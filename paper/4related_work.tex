% Related Work: You can include the related work report you submitted earlier in the semester; update it as needed.

\section{Related Work}
Our paper provide analysis of SSD architecture's influence to btree performance. We believe our work is novel in several aspects.

\subsection{Flash-optimized B+-trees}
Many schemes have been proposed for developing flash-optimized B+-trees [1, 2, 4, 5, 6, 7, 8, 9, 10, 15, 16, 19, 20].
Roh et al. [8] improved the performance of B+-Tree by utilizing the internal parallelism of the SSD.
In addition, several optimized B+-Tree designs [1, 4, 19] use similar ideas as the log-structured file system, where they employ page-level write optimizations to address the in-place update problem.
However, this log-structured updating is usually introduced at the expense of read performance.
Finally, many systems we surveyed use buffering of writes in memory to better utilize the SSD throughput, and to prevent small writes.
They delay writes in a buffer in memory so that updates can be batched together [2, 3, 7, 9, 12].
This significantly increases disk throughput and allows the system to avoid in-place updates which incur costly erase cycles that shorten SSD life.
However, this technique has negative implications for database systems, as it lengthens the time until data is persisted which weakens crash consistency.
This trade-off allows systems to achieve much greater throughput at the cost of safety.
In addition to buffering writes, some systems also keep a cache of recently and frequently accessed nodes (pages) in memory to decrease look-up time [9].\\

\subsection{Exploiting the parallelism of SSD}
That exploiting parallelism of SSD paper.